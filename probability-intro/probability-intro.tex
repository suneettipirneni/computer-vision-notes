\documentclass{article}
\usepackage{amsmath}
\usepackage{mathptmx}


\title{Introduction to probability}
\date{2023, April}
\author{Suneet Tipirneni}

\begin{document}
\maketitle

\section{Random Variables}

A random variable is a variable that represents an uncertain result. Some examples of these are \par

\begin{itemize}
	\item Flipping a coin
	\item The temperature outside
	\item Rolling a die
\end{itemize}

A random variable can be either continous or discrete. Continous variables are variables that are based on real numbers. On the other hand, a discrete variable is one with a set amount of variants. Regardless of whether a variable is discrete or continuous, each possible outcome probability must sum to $1$.

\section{Join Probability}

Joint probability refers to the combined probility of two or more random variables. These are represented like so in equation \ref{eq:joint}:

\begin{equation} \label{eq:joint}
	Pr\left( x,y \right) 
\end{equation}

Where $x$ and $y$ are two different random variables.

\end{document}
